\documentclass[twocolumn]{article}
%\documentclass{article}
\usepackage{url}
\usepackage{color}
\usepackage{multirow}
\usepackage{listings}
\usepackage{float}
\usepackage{amsfonts}
\usepackage{amssymb}
\usepackage{amsmath}
\usepackage{cite}
\usepackage{listings}
\usepackage{xcolor}
\usepackage{graphicx}
\usepackage{draftwatermark}
\SetWatermarkText{DRAFT POST}
\SetWatermarkScale{3}
\setlength\parskip{.5\baselineskip}
\author{Ryan J. Kung \\ryankung@ieee.org\\}
\title{Review: Algebraic Effect and Extensions}
\begin{document}
\maketitle
% \tableofcontents
\section{Abstract}

\section{Introduction}
Algebraic effects are an approach to computational effects based on a premise that impure behaviour arises from a set of operations such as get and set for mutable store, read and print for interactive input and output, or raise for exceptions. This naturally gives rise to handlers not only of exceptions, but of any other effect, yielding a novel concept that, amongst others, can capture stream redirection, backtracking, co-operative multi-threading, and delimited continuations\cite{intro-algebraic-effects-and-handlers}.

Operations of algebraic Effect was first introduced by Gordon Poltkin and John Power in 2002 \cite{Plotkin2003} as Algebraic Operation, based on Eugenio Moggi's work \cite{39155, MOGGI91} in 1989-1992 about logics for reasoning and proving equivalence about programs with a strong monad $T$ on a base category $C$ with finite products\cite{Plotkin2003, MOGGI91}. And made the notion system for effects such as: nondeterminism, probabilistic nodeterminism, exceptions, interactive input/output, side-effects, and continuations by identifying it with the notion of $algebraic$ operation.

\begin{quotation}
  ``Algebraic operations are, in the sense we shall make precise, a natural generalisation, from $Set$ to an arbitrary symmetric monoidal $V-category$ $C$ with cotensors.''\cite{Plotkin2003}
\end{quotation}

And effect and handeler was introduced by Plotkin and Pretnar in 2009 \cite{lmcs:705} as computaional effects that can be represented by an equational theory whose oerations produce the effect as hand, which are based on Moggi's monad work, but as a restriction on general monads, algebraic effects have many various advantages: can be freely composed, and there is a natual separation between interfaces and sematics (as handler) \cite{algebraic-effects-for-functional-programming}

\section{Implementations}

Algebraic Effect usually related to $Eff$ which support Algebraic Effects and Handlers as first class,    Algebraic Effect can also be widely use in common platform such as ECMAScript, .net, JVM, or other programming languages by using a type directed selective CPS translation\cite{algebraic-effects-for-functional-programming}.

Since Algebraic Effect and Its Operators are build based on a Monad System on category-V, it can be used for both Strong type languages line Haskell or OCaml and weak type languages like Python or ECMAScript, but there is some challenge that, a Typing algebraic effects is that inferred types became very large or difficult to understand. And for a library based implementation, it do not have full control over the runtime stack.\cite{algebraic-effects-for-functional-programming}.

\begin{itemize}
\item Koka is a function-oriented programming language that seperates pure values from side-effecting computations, where the effect of every function is automatically inferred.

\item Eff is a programming language based on the algebraic approach to computational effects, in which effects are viewed as algebraic operations and effect handlers as homomorphisms from free algebras.\cite{eff} Eff supports first-class effects and handlers through which we may easily define new computational effects, seamlessly combine existing ones, and handle them in novel ways.

\item Idris is a general purpose pure functional programming language with dependent types. From version 0.9.12 Idris includes a library for side-effect management, Effects.


\item Python Effect is a library of Python for helping write purely functional code by isolating the effects\cite{python-effect}.

\item Haskell Effect-handlers is a library for writing extensible algebraic effects and handlers with haskell\cite{effect-handlers}.
\end{itemize}
\section{Extensions}

Niki Vazou and Daan LeiJen introduced how to combine Algebraic Effect with Monads System\cite{10.1007/978-3-319-28228-2_11}.
\bibliographystyle{plain}
\bibliography{effect}{}
\end{document}